\PassOptionsToPackage{unicode=true}{hyperref} % options for packages loaded elsewhere
\PassOptionsToPackage{hyphens}{url}
%
\documentclass[]{article}
\usepackage{lmodern}
\usepackage{amssymb,amsmath}
\usepackage{ifxetex,ifluatex}
\usepackage{fixltx2e} % provides \textsubscript
\ifnum 0\ifxetex 1\fi\ifluatex 1\fi=0 % if pdftex
  \usepackage[T1]{fontenc}
  \usepackage[utf8]{inputenc}
  \usepackage{textcomp} % provides euro and other symbols
\else % if luatex or xelatex
  \usepackage{unicode-math}
  \defaultfontfeatures{Ligatures=TeX,Scale=MatchLowercase}
\fi
% use upquote if available, for straight quotes in verbatim environments
\IfFileExists{upquote.sty}{\usepackage{upquote}}{}
% use microtype if available
\IfFileExists{microtype.sty}{%
\usepackage[]{microtype}
\UseMicrotypeSet[protrusion]{basicmath} % disable protrusion for tt fonts
}{}
\IfFileExists{parskip.sty}{%
\usepackage{parskip}
}{% else
\setlength{\parindent}{0pt}
\setlength{\parskip}{6pt plus 2pt minus 1pt}
}
\usepackage{hyperref}
\hypersetup{
            pdfborder={0 0 0},
            breaklinks=true}
\urlstyle{same}  % don't use monospace font for urls
\usepackage{color}
\usepackage{fancyvrb}
\newcommand{\VerbBar}{|}
\newcommand{\VERB}{\Verb[commandchars=\\\{\}]}
\DefineVerbatimEnvironment{Highlighting}{Verbatim}{commandchars=\\\{\}}
% Add ',fontsize=\small' for more characters per line
\newenvironment{Shaded}{}{}
\newcommand{\AlertTok}[1]{\textcolor[rgb]{1.00,0.00,0.00}{\textbf{#1}}}
\newcommand{\AnnotationTok}[1]{\textcolor[rgb]{0.38,0.63,0.69}{\textbf{\textit{#1}}}}
\newcommand{\AttributeTok}[1]{\textcolor[rgb]{0.49,0.56,0.16}{#1}}
\newcommand{\BaseNTok}[1]{\textcolor[rgb]{0.25,0.63,0.44}{#1}}
\newcommand{\BuiltInTok}[1]{#1}
\newcommand{\CharTok}[1]{\textcolor[rgb]{0.25,0.44,0.63}{#1}}
\newcommand{\CommentTok}[1]{\textcolor[rgb]{0.38,0.63,0.69}{\textit{#1}}}
\newcommand{\CommentVarTok}[1]{\textcolor[rgb]{0.38,0.63,0.69}{\textbf{\textit{#1}}}}
\newcommand{\ConstantTok}[1]{\textcolor[rgb]{0.53,0.00,0.00}{#1}}
\newcommand{\ControlFlowTok}[1]{\textcolor[rgb]{0.00,0.44,0.13}{\textbf{#1}}}
\newcommand{\DataTypeTok}[1]{\textcolor[rgb]{0.56,0.13,0.00}{#1}}
\newcommand{\DecValTok}[1]{\textcolor[rgb]{0.25,0.63,0.44}{#1}}
\newcommand{\DocumentationTok}[1]{\textcolor[rgb]{0.73,0.13,0.13}{\textit{#1}}}
\newcommand{\ErrorTok}[1]{\textcolor[rgb]{1.00,0.00,0.00}{\textbf{#1}}}
\newcommand{\ExtensionTok}[1]{#1}
\newcommand{\FloatTok}[1]{\textcolor[rgb]{0.25,0.63,0.44}{#1}}
\newcommand{\FunctionTok}[1]{\textcolor[rgb]{0.02,0.16,0.49}{#1}}
\newcommand{\ImportTok}[1]{#1}
\newcommand{\InformationTok}[1]{\textcolor[rgb]{0.38,0.63,0.69}{\textbf{\textit{#1}}}}
\newcommand{\KeywordTok}[1]{\textcolor[rgb]{0.00,0.44,0.13}{\textbf{#1}}}
\newcommand{\NormalTok}[1]{#1}
\newcommand{\OperatorTok}[1]{\textcolor[rgb]{0.40,0.40,0.40}{#1}}
\newcommand{\OtherTok}[1]{\textcolor[rgb]{0.00,0.44,0.13}{#1}}
\newcommand{\PreprocessorTok}[1]{\textcolor[rgb]{0.74,0.48,0.00}{#1}}
\newcommand{\RegionMarkerTok}[1]{#1}
\newcommand{\SpecialCharTok}[1]{\textcolor[rgb]{0.25,0.44,0.63}{#1}}
\newcommand{\SpecialStringTok}[1]{\textcolor[rgb]{0.73,0.40,0.53}{#1}}
\newcommand{\StringTok}[1]{\textcolor[rgb]{0.25,0.44,0.63}{#1}}
\newcommand{\VariableTok}[1]{\textcolor[rgb]{0.10,0.09,0.49}{#1}}
\newcommand{\VerbatimStringTok}[1]{\textcolor[rgb]{0.25,0.44,0.63}{#1}}
\newcommand{\WarningTok}[1]{\textcolor[rgb]{0.38,0.63,0.69}{\textbf{\textit{#1}}}}
\setlength{\emergencystretch}{3em}  % prevent overfull lines
\providecommand{\tightlist}{%
  \setlength{\itemsep}{0pt}\setlength{\parskip}{0pt}}
\setcounter{secnumdepth}{0}
% Redefines (sub)paragraphs to behave more like sections
\ifx\paragraph\undefined\else
\let\oldparagraph\paragraph
\renewcommand{\paragraph}[1]{\oldparagraph{#1}\mbox{}}
\fi
\ifx\subparagraph\undefined\else
\let\oldsubparagraph\subparagraph
\renewcommand{\subparagraph}[1]{\oldsubparagraph{#1}\mbox{}}
\fi

% set default figure placement to htbp
\makeatletter
\def\fps@figure{htbp}
\makeatother

\title{JHSPH Interview Response}
\date{October 22, 2024}
\author{Larry D. Lee Jr.}

\begin{document}

\maketitle


\begin{quote}
Abstract: This technical note answers the questions presented in JHSPH's
Coding Challenge. In this note, I review the mathematical theory behind
combining datasets that use different stratifications and present code
that combines these datasets.
\end{quote}

\hypertarget{introduction}{%
\section{Introduction}\label{introduction}}

This technical note answers the questions presented in JHSPH's Coding
Challenge. In this note, I review the mathematical theory behind
combining datasets that use different stratifications and present code
that combines these datasets.

\hypertarget{mathematical-background}{%
\section{Mathematical Background}\label{mathematical-background}}

In this section, I briefly review the mathematical background needed to
understand how to combine stratified datasets.

\hypertarget{general-notation}{%
\subsection{General Notation}\label{general-notation}}

Let \(s : set\ T\) represent a set of entities of type \(T\). In
general, \(T\) can represent anything, however, in this technical note,
\(T\) will represent people. We will use the dot notation to reference
attributes. For example, \(s_i.age : \mathbb{R^+}\) will represent a
person's age. Whenever we have a set \(s\), \(|s|\) will denote the
number of elements within it.

\hypertarget{partitions}{%
\subsection{Partitions}\label{partitions}}

A finite \textbf{partition} of a set \(s\) is a finite set of sets
\(p : set\ (set\ T)\) such that \(s = \bigcup_{i=1}^n p_i\) and
\(\forall i, j \in \mathbb{N}, i \neq j \implies p_i \cap p_j = \emptyset\)
where \(n = |p|\). A \textbf{partition function}
\(f : T \rightarrow \mathbb{N}\) is a function that accepts a value
\(s_i\) from \(s\) and returns the index \(j\) of the subset \(p_j\)
that contains \(s_i\) in a partition \(p\).

As a special case we will often consider contiguous clopen interval
partitions (\textbf{CCIPs}) of the positive real number line
\(\mathbb{R}^+\).\footnote{We call intervals that are closed at their
  lower bound and open at their upper bound \textbf{clopen}.} We will
represent these partitions as a list of intervals in ascending order and
will represent the type of such partitions as \([Interval]\). Note that
there are no gaps between these intervals thus:

\begin{align*}
\forall p : [Interval], \forall x \in \mathbb{R}, x \ge p_0.start\ \wedge x < p_n.end \implies \exists p_i \in p, x \in p.
\end{align*}

\hypertarget{stratification}{%
\subsection{Stratification}\label{stratification}}

A \textbf{stratification} is a partition in which elements from \(s\)
are allocated to different subsets \(p\) based some property value such
as age.

In a stratification we take a set of entities of type \(T\), idenfity a
property, such as age, that spans some domain, such as \(\mathbb{R}^+\),
partition that domain \(p\), and then project entities into those
partition sets.

\hypertarget{rates}{%
\subsection{Rates}\label{rates}}

Let \(f : T \rightarrow \mathbb{B}\) represent a predicate such as the
function that accepts a person and returns true iff the person has a
given disease. We define the \textbf{frequency} of \(f\) over a set
\(s\) as:

\begin{align*}
freq\ (f, s) := |\{x \in s\ |\ f\ (x)\}|.
\end{align*}

The \textbf{rate} of a \(f\) over a set \(s\) is defined as:

\begin{align*}
rate\ (f, s) := \frac{freq\ (f, s)}{|s|}.
\end{align*}

Let \(p\) denote a finite disjoint partition of \(s\) into \(n\)
subsets, then:

\begin{align*}
freq\ (f, s) := \sum_{i=1}^n freq\ (f, p_i)
\end{align*}

and accordingly:

\begin{align}
rate\ (f, s) = \frac{\sum_{i=1}^n freq\ (f, p_i)}{|s|} = \frac{\sum_{i=1}^n rate\ (f, p_i)\ |p_i|}{\sum_{i=1}^n|p_i|}.
\end{align}

We say that a dataset gives the rates of a predicate \(f\) over a
partition \(p\) when the dataset lists the rate of \(f\) for every
element (subset of \(s\)) in \(p\).

\hypertarget{rate-ratios}{%
\subsection{Rate Ratios}\label{rate-ratios}}

Consider the case in which we have two different predicates
\(f, g: T \rightarrow \mathbb{B}\). These two functions may represent
different diseases such as HIV and Gonorrhea for instance. Often, we
want to compare their rates. Their \textbf{rate ratio} over a set \(s\)
is given by:

\begin{align*}
rateRatio (f, g, s) := \frac{rate\ (f, s)}{rate\ (g, s)} = \frac{freq\ (f, s)}{freq\ (g, s)}.
\end{align*}

Notice that the rate ratio can only be defined when the rates given
apply to the same underlying set. We say that two sets are
\textbf{compatible} if they are equal. We can calculate risk ratios
using rates over two compatible sets. Also note that the risk ratio of
\(f\) and \(g\) can be calculated by taking the ratio of their
frequencies over \(s\).

\hypertarget{refinement-and-coarsening}{%
\subsection{Refinement and Coarsening}\label{refinement-and-coarsening}}

Let \(p, q : set (set\ T)\) represent two partitions of a set \(s\). We
say that \(q\) is a \textbf{refinement} of \(p\) iff for every element
\(p_i\) in \(p\) there exist one or more elements in \(q\) that
collectively represent a partition of \(p\). In other words, there are
one or more elements in \(q\) whose union equals \(p_i\).

\begin{align*}
\forall p_i \in p, \exists r \subseteq q, p_i = \bigcup r. 
\end{align*}

Conversely, we say that \(p\) is a \textbf{coarsening} of \(q\) if \(q\)
is a refinement of \(p\).

We slightly overload our terminology by saying that \(r\) is \(p_i\)'s
\textbf{refinement} in \(q\).'

We can apply this notion to contiguous clopen interval partitions of
\(\mathbb{R}^+\). Let \(p, q : [Interval]\) represent two CCIPs of
\(\mathbb{R}^+\). We say that \(q\) \textbf{refines} \(p\) iff:

\begin{align*}
refines?\ (q, p) := \forall p_i \in p, \exists (j, n \in \mathbb{N}), p_i = \bigcup_{k=j}^{j+n} q_k.
\end{align*}

Conversely, we say that \(p\) \textbf{coarsens} \(q\). We say that \(q\)
is \textbf{more granular} than \(p\) when \(q\) refines \(p\).

\hypertarget{rates-over-coarsening}{%
\subsection{Rates over Coarsening}\label{rates-over-coarsening}}

Let \(p, \pi : [Interval]\) be two CCIPs where \(\pi\) generalizes
\(p\). Let's assume furthermore that we have rates for some predicate
\(f\) over \(p\). We can use (1) to calculate the corresponding rates of
\(f\) over \(\pi\).

If we know the proportion \(prop (p_i) := |p_i|/|s|\) of entities in the
underlying set \(s\) that are grouped within each interval \(p_i\), we
can rewrite (1) as:

\begin{align*}
rate\ (f, \pi) = \sum_{i = 1}^n rate\ (f, p_i)\ prop\ (p_i).
\end{align*}

\hypertarget{rate-ratios-over-coarsening}{%
\subsection{Rate Ratios over
Coarsening}\label{rate-ratios-over-coarsening}}

We can calculate the rate ratios over two partitions \(p\) and \(q\)
when \(q\) is a refinement of \(p\). The rate ratios will be defined for
elements (sets of \(s\)) defined by the less granular partition \(p\).
For every element \(p_i\) in \(p\), let \(r_i\) represent \(p_i\)'s
refinement in \(q\). Then the risk ratio of \(f\) and \(g\) over \(p\)
is given by:

\begin{align*}
rateRatio (f, g, p_i, r_i) := \frac{rate\ (f, p_i)}{rate\ (g, r_i)} = \frac{freq\ (f, p_i)}{freq\ (g, r_i)}.
\end{align*}

\hypertarget{generalization}{%
\subsection{Generalization}\label{generalization}}

Let's say that \(p, q, \pi : [Interval]\) represent three CCIPs of
\(\mathbb{R}^+\). We say that \(\pi\) \textbf{generalizes} \(p\) and
\(q\) iff:

\begin{align*}
generalizes?\ (\pi, p, q) := refines? (\pi, p) \wedge refines?\ (\pi, q).
\end{align*}

Notice that the intervals in \(\pi\) start and end at boundary points
shared by both \(p\) and \(q\).

We say that \(\pi\) is the \textbf{most granular generalization} of
\(p\) and \(q\) iff \(\pi\) refines every other generalization of \(p\)
and \(q\).

\hypertarget{rounding-errors}{%
\subsection{Rounding Errors}\label{rounding-errors}}

When working with real world datasets we have to account for the
accumulation of rounding errors when combining rates for partitions.
Imagine that we have an interval and a rate of 0.53. If we extended the
accuracy of the rate to four significant digits, we could find that the
true rate was anything from 0.5250 to 0.5349.

\hypertarget{regression-estimates}{%
\subsection{Regression Estimates}\label{regression-estimates}}

In this technical note, we've considered datasets that use different
data stratifications and discussed ways to define new stratifications
that we can use to combine frequencies and rates. This approach produces
the most accurate results across composite datasets. However, this
discussion would be incomplete without discussing the use of regression
analysis to estimate rates across populations.

Consider the case in which we have a set \(s\) of entities of type
\(T\). \(T\) in this case will represent people. We can stratify the
people in \(s\) using a partition \(p\). For instance, we can stratify
people by age where each age group corresponds to a five year interval.
Significantly however we can gradually refine our partition by using a
series of increasingly shorter intervals. For instance, we can refine
\(p\) into 1 year intervals, then 1 month intervals, then 1 day
intervals, etc. Theoretically, we can continue this refinement process
until we reach a limit of a continuous time distribution.

Let's say that we have a set \(s\) of people (entities of type \(T\)).
We stratify this population by age using equal intervals (for example 1
year intervals) \(p_i\) and then calculate the rates of some condition
\(f\) for each subgroup.

\hypertarget{answers}{%
\section{Answers}\label{answers}}

The Interview challenge opens with the following questions:

\begin{quote}
PART 1: Using these four datasets, calculate the following quantities,
or if they cannot be calculated exactly, explain why

\begin{enumerate}
\def\labelenumi{\arabic{enumi}.}
\item
  The ratio of gonorrhea rate to HIV diagnosis rate among people ages
  25-44, and among people ages 45 or greater, in 2020
\item
  The ratio of heroin use rate to HIV diagnosis rate among people ages
  12 and older in 2020
\end{enumerate}
\end{quote}

I've included a code package that includes code written in both R and
OCaml to answer these questions. We can calculate the frequency of
Gonorrhea and HIV for each of the age ranges included in their datasets.
We can then calculate the sum of people in both datasets who have these
conditions in the 25-44 age cohorts. Dividing these sums we find that:

\begin{enumerate}
\def\labelenumi{\arabic{enumi}.}
\item
  (a) The ratio of gonorrhea rate to HIV diagnosis rate among people
  aged 25-44 is \textbf{19.8}.
\end{enumerate}

Note that this calculation is approximate but accurate to 3 significant
digits. We relied on underlying rate estimates for Gonorrhea and HIV.
Both of these datasets rounded rate estimates to the nearest tenth of
1/100,000th. Thus, given an estimate such as 25.7 per 100,000, the true
rate is anywhere between 25.65 and 25.74. These errors can compound
enough to shift our estimate but not by enough to alter the figure
given. If we extend our estimate to four significant digits the
potential error ranges from 19.82 to 19.84.

\begin{enumerate}
\def\labelenumi{\arabic{enumi}.}
\item
  (b) The ratio of gonorrhea rate to HIV diagnosis rate among people
  aged 45 and older is \textbf{6.7}
\end{enumerate}

This calculation is accurate to the number of digits shown. However,
given the rounding error implicit in the underlying rate estimates for
Gonorrhea and HIV, the true ratio could range from 6.72 to 6.75.

\begin{enumerate}
\def\labelenumi{\arabic{enumi}.}
\item
  We cannot accurately estimate the ratio of heroin use rate to HIV
  diagnosis rate for people ages 12 and older.
\end{enumerate}

The reason we cannot estimate this datum is because the age partitions
used by the HIV rate dataset and the population age distribution are not
compatible (as defined above). The best that we can do is to use some
form of curve fitting or interpolation to estimate population density
over age and then use this to estimate the number of people falling
within the age ranges used by the heroin dataset.

The interview challenge continues by giving a specification for a
function that "... takes three datasets as inputs (two with rates per
population and one with population) and returns the ratio of these rates
by the most granular age brackets possible". The \texttt{getRateRatio}
function defined in the code sample given in Appendix 1 implements this
function specification.

\hypertarget{appendix-1-r-code-sample}{%
\section{Appendix 1: R Code Sample}\label{appendix-1-r-code-sample}}

\begin{Shaded}
\begin{Highlighting}[]
\CommentTok{# This package contains definitions that can be used to represent and combine}
\CommentTok{# stratified datasets.}
\CommentTok{#}
\CommentTok{# It defines a datatype named interval that represents half open intervals of}
\CommentTok{# the real number line. These intervals can be used to partition the real number}
\CommentTok{# line into a finite number of intervals. A population can be divided across}
\CommentTok{# these intervals and information about the subpopulation associated with each}
\CommentTok{# group can be added.}
\CommentTok{#}
\CommentTok{# At the end of this file, we use these definitions to describe the prevalence}
\CommentTok{# of HIV and gonorrhea across Baltimore City.}

\NormalTok{UNBOUND=}\KeywordTok{quote}\NormalTok{(unbound)}

\CommentTok{# Constructs an annotated interval}
\CommentTok{# These objects represent the closed-open intervals of the postive real number}
\CommentTok{# line.}
\CommentTok{# Annotated partitions are represented by list of interval objects}
\NormalTok{interval <-}\StringTok{ }\ControlFlowTok{function}\NormalTok{ (info, start, }\DataTypeTok{end =}\NormalTok{ UNBOUND) \{}
  \KeywordTok{list}\NormalTok{ (}\DataTypeTok{info=}\NormalTok{info, }\DataTypeTok{start=}\NormalTok{start, }\DataTypeTok{end=}\NormalTok{end)}
\NormalTok{\}}

\CommentTok{# Accepts two real numbers x and y who may be "unbound" and compares them.}
\NormalTok{boundCompare <-}\StringTok{ }\ControlFlowTok{function}\NormalTok{ (x, y) \{}
  \ControlFlowTok{if}\NormalTok{ (x }\OperatorTok{==}\StringTok{ }\NormalTok{y) \{ }
    \KeywordTok{return}\NormalTok{ (}\DecValTok{0}\NormalTok{)}
\NormalTok{  \} }\ControlFlowTok{else} \ControlFlowTok{if}\NormalTok{ (x }\OperatorTok{<}\StringTok{ }\NormalTok{y }\OperatorTok{||}\StringTok{ }\NormalTok{y }\OperatorTok{==}\StringTok{ }\NormalTok{UNBOUND) \{}
    \KeywordTok{return}\NormalTok{ (}\OperatorTok{-}\DecValTok{1}\NormalTok{)}
\NormalTok{  \} }\ControlFlowTok{else} \ControlFlowTok{if}\NormalTok{ (x }\OperatorTok{>}\StringTok{ }\NormalTok{y }\OperatorTok{||}\StringTok{ }\NormalTok{x }\OperatorTok{==}\StringTok{ }\NormalTok{UNBOUND) \{}
    \KeywordTok{return}\NormalTok{ (}\DecValTok{1}\NormalTok{)}
\NormalTok{  \}}
\NormalTok{\}}

\CommentTok{# Accepts two real numbers x and y who may be "unbound" and returns true iff x}
\CommentTok{# is less than or equal to y.}
\NormalTok{boundLte <-}\StringTok{ }\ControlFlowTok{function}\NormalTok{ (x, y) }\KeywordTok{boundCompare}\NormalTok{ (x, y) }\OperatorTok{<=}\StringTok{ }\DecValTok{0}

\CommentTok{# Accepts two real numbers who may be "unbound" and returns the smaller of them.}
\NormalTok{boundMin <-}\StringTok{ }\ControlFlowTok{function}\NormalTok{ (x, y) \{}
  \ControlFlowTok{if}\NormalTok{ (x }\OperatorTok{==}\StringTok{ }\NormalTok{UNBOUND) \{}
\NormalTok{    y}
\NormalTok{  \} }\ControlFlowTok{else} \ControlFlowTok{if}\NormalTok{ (y }\OperatorTok{==}\StringTok{ }\NormalTok{UNBOUND) \{}
\NormalTok{    x}
\NormalTok{  \} }\ControlFlowTok{else}\NormalTok{ \{}
    \KeywordTok{min}\NormalTok{ (x, y)}
\NormalTok{  \}}
\NormalTok{\}}

\CommentTok{# Accepts three arguments:}
\CommentTok{#}
\CommentTok{# @param f a function that accepts two contiguous series of annotated intervals}
\CommentTok{#   that span the same range of the positive real number line and merges them}
\CommentTok{# @param xs, an annotated partition}
\CommentTok{# @param ys, an annotated partition}
\CommentTok{#}
\CommentTok{# And returns a new annotated partition based on the most granular intervals}
\CommentTok{# spanned by xs and ys whose annotations are derived from f.}
\NormalTok{rebase <-}\StringTok{ }\ControlFlowTok{function}\NormalTok{ (f, xs, ys) \{}
\NormalTok{  zs <-}\StringTok{ }\KeywordTok{list}\NormalTok{ ()}
\NormalTok{  i <-}\StringTok{ }\DecValTok{1}
\NormalTok{  j <-}\StringTok{ }\DecValTok{1}
  \ControlFlowTok{while}\NormalTok{ (i }\OperatorTok{<=}\StringTok{ }\KeywordTok{length}\NormalTok{ (xs) }\OperatorTok{&&}\StringTok{ }\NormalTok{j }\OperatorTok{<=}\StringTok{ }\KeywordTok{length}\NormalTok{ (ys)) \{}
\NormalTok{    x <-}\StringTok{ }\NormalTok{xs[[i]]}
\NormalTok{    y <-}\StringTok{ }\NormalTok{ys[[j]]}
    \ControlFlowTok{if}\NormalTok{ (x}\OperatorTok{$}\NormalTok{start }\OperatorTok{==}\StringTok{ }\NormalTok{y}\OperatorTok{$}\NormalTok{start) \{}
\NormalTok{      xInfos <-}\StringTok{ }\KeywordTok{c}\NormalTok{ ()}
\NormalTok{      yInfos <-}\StringTok{ }\KeywordTok{c}\NormalTok{ ()}
\NormalTok{      start <-}\StringTok{ }\NormalTok{x}\OperatorTok{$}\NormalTok{start}
      \ControlFlowTok{while}\NormalTok{ (i }\OperatorTok{<=}\StringTok{ }\KeywordTok{length}\NormalTok{ (xs) }\OperatorTok{&&}\StringTok{ }\NormalTok{j }\OperatorTok{<=}\StringTok{ }\KeywordTok{length}\NormalTok{ (ys)) \{}
\NormalTok{        x <-}\StringTok{ }\NormalTok{xs[[i]]}
\NormalTok{        y <-}\StringTok{ }\NormalTok{ys[[j]]}
        \ControlFlowTok{if}\NormalTok{ (}
          \KeywordTok{boundCompare}\NormalTok{ (x}\OperatorTok{$}\NormalTok{end, y}\OperatorTok{$}\NormalTok{end) }\OperatorTok{==}\StringTok{ }\DecValTok{0} \OperatorTok{||}
\StringTok{          }\NormalTok{(y}\OperatorTok{$}\NormalTok{end }\OperatorTok{==}\StringTok{ }\NormalTok{UNBOUND }\OperatorTok{&&}\StringTok{ }\NormalTok{i }\OperatorTok{==}\StringTok{ }\KeywordTok{length}\NormalTok{ (xs)) }\OperatorTok{||}
\StringTok{          }\NormalTok{(x}\OperatorTok{$}\NormalTok{end }\OperatorTok{==}\StringTok{ }\NormalTok{UNBOUND }\OperatorTok{&&}\StringTok{ }\NormalTok{j }\OperatorTok{==}\StringTok{ }\KeywordTok{length}\NormalTok{ (ys))}
\NormalTok{        ) \{}
\NormalTok{          zs <-}\StringTok{ }\KeywordTok{append}\NormalTok{ (zs,}
            \KeywordTok{list}\NormalTok{ (}\KeywordTok{interval}\NormalTok{ (}
              \DataTypeTok{info=}\KeywordTok{f}\NormalTok{ (}
                \KeywordTok{c}\NormalTok{ (xInfos, x}\OperatorTok{$}\NormalTok{info),}
                \KeywordTok{c}\NormalTok{ (yInfos, y}\OperatorTok{$}\NormalTok{info)}
\NormalTok{              ),}
              \DataTypeTok{start=}\NormalTok{start,}
              \DataTypeTok{end=}\KeywordTok{boundMin}\NormalTok{ (x}\OperatorTok{$}\NormalTok{end, y}\OperatorTok{$}\NormalTok{end)}
\NormalTok{          )))}
\NormalTok{          i <-}\StringTok{ }\NormalTok{i }\OperatorTok{+}\StringTok{ }\DecValTok{1}
\NormalTok{          j <-}\StringTok{ }\NormalTok{j }\OperatorTok{+}\StringTok{ }\DecValTok{1}
          \ControlFlowTok{break}
\NormalTok{        \} }\ControlFlowTok{else} \ControlFlowTok{if}\NormalTok{ (}\KeywordTok{boundCompare}\NormalTok{ (x}\OperatorTok{$}\NormalTok{end, y}\OperatorTok{$}\NormalTok{end) }\OperatorTok{==}\StringTok{ }\DecValTok{-1}\NormalTok{) \{}
\NormalTok{          xInfos <-}\StringTok{ }\KeywordTok{c}\NormalTok{ (xInfos, x}\OperatorTok{$}\NormalTok{info)}
\NormalTok{          i <-}\StringTok{ }\NormalTok{i }\OperatorTok{+}\StringTok{ }\DecValTok{1}
\NormalTok{        \} }\ControlFlowTok{else}\NormalTok{ \{ }\CommentTok{# x$end > y$end}
\NormalTok{          yInfos <-}\StringTok{ }\KeywordTok{c}\NormalTok{ (yInfos, y}\OperatorTok{$}\NormalTok{info)}
\NormalTok{          j <-}\StringTok{ }\NormalTok{j }\OperatorTok{+}\StringTok{ }\DecValTok{1}
\NormalTok{        \}}
\NormalTok{      \}}
\NormalTok{    \} }\ControlFlowTok{else} \ControlFlowTok{if}\NormalTok{ (x}\OperatorTok{$}\NormalTok{start }\OperatorTok{<}\StringTok{ }\NormalTok{y}\OperatorTok{$}\NormalTok{start) \{}
\NormalTok{      i <-}\StringTok{ }\NormalTok{i }\OperatorTok{+}\StringTok{ }\DecValTok{1}
\NormalTok{    \} }\ControlFlowTok{else}\NormalTok{ \{ }\CommentTok{# x$start > y$start}
\NormalTok{      j <-}\StringTok{ }\NormalTok{j }\OperatorTok{+}\StringTok{ }\DecValTok{1}
\NormalTok{    \}}
\NormalTok{  \}}
\NormalTok{  zs}
\NormalTok{\}}

\CommentTok{# Accepts two partitions:}
\CommentTok{# @param ps a partition that represents the number of people in a population who}
\CommentTok{#   fall within a contiguous set of age ranges}
\CommentTok{# @param rs a partition that represents the proportion of people within a}
\CommentTok{#   contiguous set of age ranges who have some condition}
\CommentTok{# and returns a partition that gives the absolute number of people who have the}
\CommentTok{# condition within the most granular set of age ranges spanned by both}
\CommentTok{# ps and rs.}
\CommentTok{# @warn }
\NormalTok{getFrequency <-}\StringTok{ }\ControlFlowTok{function}\NormalTok{ (ps, rs) \{}
  \KeywordTok{rebase}\NormalTok{ (}
    \ControlFlowTok{function}\NormalTok{ (sizes, rates) \{}
      \ControlFlowTok{if}\NormalTok{ (}\KeywordTok{length}\NormalTok{ (rates) }\OperatorTok{!=}\StringTok{ }\DecValTok{1}\NormalTok{) \{}
        \KeywordTok{stop}\NormalTok{ (}\StringTok{"Error: the population partition is not a }\CharTok{\textbackslash{}"}\StringTok{refinement}\CharTok{\textbackslash{}"}\StringTok{ of the rate partition"}\NormalTok{)}
\NormalTok{      \}}
\NormalTok{      rates[}\DecValTok{1}\NormalTok{]}\OperatorTok{*}\KeywordTok{sum}\NormalTok{ (sizes)}
\NormalTok{    \}, ps, rs}
\NormalTok{  )}
\NormalTok{\}}

\CommentTok{# Accepts three arguments:}
\CommentTok{# @start the age range lower bound}
\CommentTok{# @end the age range upper bound}
\CommentTok{# @freqs the number of people who have a given condition for a set of age ranges}
\CommentTok{# and returns the total number of people who have the given condition between}
\CommentTok{# the start and end age range (inclusive).}
\NormalTok{getFreqSum <-}\StringTok{ }\ControlFlowTok{function}\NormalTok{ (start, end, freqs) \{}
\NormalTok{  sum <-}\StringTok{ }\DecValTok{0}
  \ControlFlowTok{for}\NormalTok{ (freq }\ControlFlowTok{in}\NormalTok{ freqs) \{}
    \ControlFlowTok{if}\NormalTok{ (start }\OperatorTok{<=}\StringTok{ }\NormalTok{freq}\OperatorTok{$}\NormalTok{start }\OperatorTok{&&}\StringTok{ }\KeywordTok{boundLte}\NormalTok{ (freq}\OperatorTok{$}\NormalTok{end, end)) \{}
\NormalTok{      sum <-}\StringTok{ }\NormalTok{sum }\OperatorTok{+}\StringTok{ }\NormalTok{freq}\OperatorTok{$}\NormalTok{info}
\NormalTok{    \}}
\NormalTok{  \}}
\NormalTok{  sum}
\NormalTok{\}}

\CommentTok{# Accepts two partitions:}
\CommentTok{# @param xs a partition that gives the number of people within a contiguous set}
\CommentTok{#   of age ranges who have condition x}
\CommentTok{# @param ys a partition that gives the number of people within a contiguous set}
\CommentTok{#   of age ranges who have condition y}
\CommentTok{# and returns the rate ratios of the two conditions over the most granular}
\CommentTok{# partition spanned by both xs and ys.}
\CommentTok{# Note that xs and ys do not have use the same partition intervals.}
\NormalTok{getFreqRateRatio <-}\StringTok{ }\ControlFlowTok{function}\NormalTok{ (xs, ys) \{}
  \KeywordTok{rebase}\NormalTok{ (}
    \ControlFlowTok{function}\NormalTok{ (freqs0, freqs1) \{}
      \KeywordTok{sum}\NormalTok{ (freqs0)}\OperatorTok{/}\KeywordTok{sum}\NormalTok{ (freqs1)}
\NormalTok{    \}, xs, ys}
\NormalTok{  )}
\NormalTok{\}}

\CommentTok{# Accepts three partitions:}
\CommentTok{# @param ps a partition that gives the number of people who fall into a}
\CommentTok{#   contiguous set of real intervals}
\CommentTok{# @param xs a partition that gives the proportion of people within a range of}
\CommentTok{#   intervals who have a condition x}
\CommentTok{# @param ys a partition that gives the proportion of people within a range of}
\CommentTok{#   intervals who have a condition y}
\CommentTok{# and returns the rate ratio of the number of people who have condition x and y}
\CommentTok{# over the most granular common partition that spans xs and ys.}
\NormalTok{getRateRatio <-}\StringTok{ }\ControlFlowTok{function}\NormalTok{ (ps, xs, ys) \{}
\NormalTok{  xFreq <-}\StringTok{ }\KeywordTok{getFrequency}\NormalTok{ (ps, xs)}
\NormalTok{  yFreq <-}\StringTok{ }\KeywordTok{getFrequency}\NormalTok{ (ps, ys)}
  \KeywordTok{getFreqRateRatio}\NormalTok{ (xFreq, yFreq)}
\NormalTok{\}}

\CommentTok{# A partition listing the number of Baltimore residents who's ages fall within}
\CommentTok{# a contiguous set of age ranges.}
\NormalTok{population =}\StringTok{ }\KeywordTok{list}\NormalTok{ (}
  \KeywordTok{interval}\NormalTok{ (}\DecValTok{36355}\NormalTok{, }\DecValTok{0}\NormalTok{, }\DecValTok{4}\NormalTok{),}
  \KeywordTok{interval}\NormalTok{ (}\DecValTok{33773}\NormalTok{, }\DecValTok{5}\NormalTok{, }\DecValTok{9}\NormalTok{),}
  \KeywordTok{interval}\NormalTok{ (}\DecValTok{33590}\NormalTok{, }\DecValTok{10}\NormalTok{, }\DecValTok{14}\NormalTok{),}
  \KeywordTok{interval}\NormalTok{ (}\DecValTok{33872}\NormalTok{, }\DecValTok{15}\NormalTok{, }\DecValTok{19}\NormalTok{),}
  \KeywordTok{interval}\NormalTok{ (}\DecValTok{37183}\NormalTok{, }\DecValTok{20}\NormalTok{, }\DecValTok{24}\NormalTok{),}
  \KeywordTok{interval}\NormalTok{ (}\DecValTok{53357}\NormalTok{, }\DecValTok{25}\NormalTok{, }\DecValTok{29}\NormalTok{),}
  \KeywordTok{interval}\NormalTok{ (}\DecValTok{54804}\NormalTok{, }\DecValTok{30}\NormalTok{, }\DecValTok{34}\NormalTok{),}
  \KeywordTok{interval}\NormalTok{ (}\DecValTok{43408}\NormalTok{, }\DecValTok{35}\NormalTok{, }\DecValTok{39}\NormalTok{),}
  \KeywordTok{interval}\NormalTok{ (}\DecValTok{34271}\NormalTok{, }\DecValTok{40}\NormalTok{, }\DecValTok{44}\NormalTok{),}
  \KeywordTok{interval}\NormalTok{ (}\DecValTok{30273}\NormalTok{, }\DecValTok{45}\NormalTok{, }\DecValTok{49}\NormalTok{),}
  \KeywordTok{interval}\NormalTok{ (}\DecValTok{33423}\NormalTok{, }\DecValTok{50}\NormalTok{, }\DecValTok{54}\NormalTok{),}
  \KeywordTok{interval}\NormalTok{ (}\DecValTok{37639}\NormalTok{, }\DecValTok{55}\NormalTok{, }\DecValTok{59}\NormalTok{),}
  \KeywordTok{interval}\NormalTok{ (}\DecValTok{36895}\NormalTok{, }\DecValTok{60}\NormalTok{, }\DecValTok{64}\NormalTok{),}
  \KeywordTok{interval}\NormalTok{ (}\DecValTok{29868}\NormalTok{, }\DecValTok{65}\NormalTok{, }\DecValTok{69}\NormalTok{),}
  \KeywordTok{interval}\NormalTok{ (}\DecValTok{22486}\NormalTok{, }\DecValTok{70}\NormalTok{, }\DecValTok{74}\NormalTok{),}
  \KeywordTok{interval}\NormalTok{ (}\DecValTok{13910}\NormalTok{, }\DecValTok{75}\NormalTok{, }\DecValTok{79}\NormalTok{),}
  \KeywordTok{interval}\NormalTok{ (}\DecValTok{8977}\NormalTok{, }\DecValTok{80}\NormalTok{, }\DecValTok{84}\NormalTok{),}
  \KeywordTok{interval}\NormalTok{ (}\DecValTok{9073}\NormalTok{, }\DecValTok{85}\NormalTok{)}
\NormalTok{)}

\CommentTok{# A partition listing the rates of HIV amongst Baltimore residents falling}
\CommentTok{# within certain age ranges.}
\NormalTok{hivRates =}\StringTok{ }\KeywordTok{list}\NormalTok{ (}
  \KeywordTok{interval}\NormalTok{ (}\FloatTok{45.6e-5}\NormalTok{, }\DecValTok{13}\NormalTok{, }\DecValTok{24}\NormalTok{),}
  \KeywordTok{interval}\NormalTok{ (}\FloatTok{53.6e-5}\NormalTok{, }\DecValTok{25}\NormalTok{, }\DecValTok{34}\NormalTok{),}
  \KeywordTok{interval}\NormalTok{ (}\FloatTok{46.6e-5}\NormalTok{, }\DecValTok{35}\NormalTok{, }\DecValTok{44}\NormalTok{),}
  \KeywordTok{interval}\NormalTok{ (}\FloatTok{26.7e-5}\NormalTok{, }\DecValTok{45}\NormalTok{, }\DecValTok{54}\NormalTok{),}
  \KeywordTok{interval}\NormalTok{ (}\FloatTok{5.4e-5}\NormalTok{, }\DecValTok{55}\NormalTok{, }\DecValTok{64}\NormalTok{),}
  \KeywordTok{interval}\NormalTok{ (}\FloatTok{30.4e-5}\NormalTok{, }\DecValTok{65}\NormalTok{)}
\NormalTok{)}

\CommentTok{#  A partition list the rates of Gonorrhea amongst Baltimore residents falling}
\CommentTok{# within certain age ranges.}
\NormalTok{gonorrheaRates =}\StringTok{ }\KeywordTok{list}\NormalTok{ (}
  \KeywordTok{interval}\NormalTok{ (}\FloatTok{25.7e-5}\NormalTok{, }\DecValTok{0}\NormalTok{, }\DecValTok{14}\NormalTok{),}
  \KeywordTok{interval}\NormalTok{ (}\FloatTok{2021.6e-5}\NormalTok{, }\DecValTok{15}\NormalTok{, }\DecValTok{19}\NormalTok{),}
  \KeywordTok{interval}\NormalTok{ (}\FloatTok{2647.6e-5}\NormalTok{, }\DecValTok{20}\NormalTok{, }\DecValTok{24}\NormalTok{),}
  \KeywordTok{interval}\NormalTok{ (}\FloatTok{1477.8e-5}\NormalTok{, }\DecValTok{25}\NormalTok{, }\DecValTok{29}\NormalTok{),}
  \KeywordTok{interval}\NormalTok{ (}\FloatTok{1047.3e-5}\NormalTok{, }\DecValTok{30}\NormalTok{, }\DecValTok{34}\NormalTok{),}
  \KeywordTok{interval}\NormalTok{ (}\FloatTok{773.0e-5}\NormalTok{, }\DecValTok{35}\NormalTok{, }\DecValTok{39}\NormalTok{),}
  \KeywordTok{interval}\NormalTok{ (}\FloatTok{490.3e-5}\NormalTok{, }\DecValTok{40}\NormalTok{, }\DecValTok{44}\NormalTok{),}
  \KeywordTok{interval}\NormalTok{ (}\FloatTok{298.6e-5}\NormalTok{, }\DecValTok{45}\NormalTok{, }\DecValTok{54}\NormalTok{),}
  \KeywordTok{interval}\NormalTok{ (}\FloatTok{139.5e-5}\NormalTok{, }\DecValTok{55}\NormalTok{, }\DecValTok{64}\NormalTok{),}
  \KeywordTok{interval}\NormalTok{ (}\FloatTok{23.6e-5}\NormalTok{, }\DecValTok{65}\NormalTok{)}
\NormalTok{)}

\CommentTok{# A partition listing the rates of heroin use amongst Baltimore residents}
\CommentTok{# falling within certain age ranges}
\CommentTok{# Note: The handout appears to have a typo in which heroin usage rates are}
\CommentTok{# reported as "rates per 100,000"}
\NormalTok{heroinRates =}\StringTok{ }\KeywordTok{list}\NormalTok{ (}
  \KeywordTok{interval}\NormalTok{ (}\FloatTok{0.00112}\NormalTok{, }\DecValTok{12}\NormalTok{, }\DecValTok{17}\NormalTok{),}
  \KeywordTok{interval}\NormalTok{ (}\FloatTok{0.005426598}\NormalTok{, }\DecValTok{18}\NormalTok{, }\DecValTok{25}\NormalTok{),}
  \KeywordTok{interval}\NormalTok{ (}\FloatTok{0.009849983}\NormalTok{, }\DecValTok{26}\NormalTok{)}
\NormalTok{)}

\CommentTok{# The number of people within certain age ranges who have HIV }
\NormalTok{hivFreq =}\StringTok{ }\KeywordTok{getFrequency}\NormalTok{ (population, hivRates)}

\CommentTok{# The number of people within certain age ranges who have Gonorrhea}
\NormalTok{gonorrheaFreq =}\StringTok{ }\KeywordTok{getFrequency}\NormalTok{ (population, gonorrheaRates)}

\CommentTok{# The number of people within certain age ranges who use heroin}
\NormalTok{heroinFreq =}\StringTok{ }\KeywordTok{getFrequency}\NormalTok{ (population, heroinRates)}

\CommentTok{# The rate ratio of gonorrhea and HIV for people aged 25 to 44}
\CommentTok{# Note: the answer to question 1.(a)}
\NormalTok{gonorrheaHivFreq2544 =}
\StringTok{  }\KeywordTok{getFreqSum}\NormalTok{ (}\DecValTok{25}\NormalTok{, }\DecValTok{44}\NormalTok{, gonorrheaFreq)}\OperatorTok{/}
\StringTok{  }\KeywordTok{getFreqSum}\NormalTok{ (}\DecValTok{25}\NormalTok{, }\DecValTok{44}\NormalTok{, hivFreq)}

\CommentTok{# The rate ratio of gonorrhea and HIV for people aged 45 and older}
\CommentTok{# Note: the answer to question 1.(b)}
\NormalTok{gonorrheaHivfreq =}
\StringTok{  }\KeywordTok{getFreqSum}\NormalTok{ (}\DecValTok{45}\NormalTok{, UNBOUND, gonorrheaFreq)}\OperatorTok{/}
\StringTok{  }\KeywordTok{getFreqSum}\NormalTok{ (}\DecValTok{45}\NormalTok{, UNBOUND, hivFreq)}
\end{Highlighting}
\end{Shaded}

\end{document}
